\section{Introduction}
\label{sec:intro}

When the term ``animation'' or ``motion graphics'' is mentioned, people often think of the process of creating animated films or cartoons. This usually refers to hand-drawn animations or, more recently, computer-generated imagery (CGI) and visual effects. However, the terms "animation" and "motion graphics" have much broader meanings.

As Adobe, an industry leader in digital media, defines it: "Motion graphics are essentially 'graphics with movement'." \cite{adobe-motion-graphics} This definition aligns more closely with the theme of this thesis. What are procedural animations? What systems can be used to create them? What are rule-based animation systems, and what emergent behaviors can be achieved that are unforeseen by the designer?

The premise is that everything that moves on the screen is animation, and everything that is animated can be considered motion graphics. This thesis will delve into the different algorithms, aspects, use cases, and applications of these systems, and how they are used throughout the digital world, even though we might not notice them at first glance.

The basic hypothesis is that we can create complex movements, behaviors, and animations by defining simple rules and systems. In other words, the less artistic control we have over the final result, the more we can rely on the system to generate interesting, natural-looking motion.

\section{Background}
\label{sec:background}

To understand the concepts discussed in this thesis, it is essential to have a basic understanding of vectors, matrices, affine transformations, and some principles of computer graphics. An excellent resource for this can be found in the book "Learn OpenGL - Graphics Programming" by Joey de Vries \cite{learn-opengl}. This book provides a comprehensive introduction to the mathematical foundations of computer graphics, which are crucial for understanding procedural animations and motion graphics.

My recommendation is to read sections 8 through 10 of the book, which covers all the necessary mathematical elements required to proceed.

In addition, it is beneficial to have a basic understanding of programming, especially in JavaScript, as this thesis will include code examples and implementations in this language. I will try to explain everything in mathematical terms first, and then provide the code examples to illustrate the concepts. This approach will help bridge the gap between theory and practice, making it easier to grasp the underlying principles of procedural animations and motion graphics.
