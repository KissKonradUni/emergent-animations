\section{Introduction}
\label{sec:intro}

When the term "animation" or "motion graphics" is mentioned, people often think of the process of creating animated films or cartoons. This usually refers to hand-drawn animations or, more recently, computer-aided visual effects. However, the terms "animation" and "motion graphics" have a much broader meaning.

As Adobe, an industry leader in digital media, defines it: "Motion graphics are essentially 'graphics with movement'." \cite{adobe-motion-graphics} This definition is much more in line with the theme of this thesis. What are procedural animations? What systems can be used to create them? What are rule-based animation systems, and what emergent behaviors can be achieved that are not foreseen by the designer? 

The baseline is that everything that moves on the screen is animation, and everything that is animated can be considered motion graphics. This thesis will delve into the different algorithms, aspects, use cases, and applications of these systems, and how they are used everywhere in the digital world, even though we might not notice them at first glance.
