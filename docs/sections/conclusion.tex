\pagebreak

\section{Conclusion}
\label{sec:conclusion}

In this thesis, I have explored a variety of animation techniques, ranging from simple functional and interpolation-based methods to more complex systems such as rule-based algorithms, blend trees, and physics simulations. While not every technique was measured in detail, I focused on representative examples to illustrate the strengths and limitations of each approach.

The results support the main hypothesis: techniques that are easily created and controlled, such as interpolation and functional animation, offer precise and predictable motion. In contrast, emergent systems—those based on rules, data, or simulation—tend to produce more natural and lifelike movement, often at the expense of direct control.

Ultimately, the choice of technique depends on the desired balance between control and realism. By understanding the properties and trade-offs of each method, it is possible to design animation systems that are both expressive and efficient, tailored to the needs of the application. The combination of multiple techniques often yields the best results, leveraging the strengths of each to create rich and dynamic animations.
