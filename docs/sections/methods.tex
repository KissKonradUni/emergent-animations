\section{Methods}
\label{sec:methods}

Before diving into the the complex implementations and algorithms, it is essential we visualize the core concepts mentioned in the previous section.

\begin{Note} 
    All of these visualizations are created using my own TypeScript implementation of a simple graphics library built on top of the HTML5 Canvas API. The code for this library is available in the repository of this thesis, and the visualizations can be viewed in the browser here: \url{https://kisskonraduni.github.io/emergent-animations/examples}.
\end{Note}

\subsection{Visualizing Coordinate Systems and Transformations}
\label{sec:visualizing-coordinate-systems}

From here on it is recommended to have the examples open on a nearby device, as the visualizations will be referenced throughout the thesis. I will provide still images for most of the examples, but as they are 'motion graphics', it is best to see them in action. For starters, let's look at the different spaces mentioned in one of \ref{sec:spaces}'s examples. 

\setlength{\columnsep}{1.5cm}
\setlength{\columnseprule}{0.4pt}

\begin{multicols}{3}
    
    \center\textbf{World Space}

\columnbreak

    \center\textbf{Object Space}

\columnbreak

    \center\textbf{Screen Space}

\end{multicols}

\pagebreak
